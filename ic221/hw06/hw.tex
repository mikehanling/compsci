\documentclass{article}[9pt]

\usepackage{listings}
\usepackage{fullpage}
\usepackage{textcomp}
\usepackage{mdframed}

\lstset{ %
  basicstyle=\ttfamily\scriptsize,
  commentstyle=\ttfamily\scriptsize\emph,
  upquote=true,
  framerule=1.25pt,
  breaklines=true,
  showstringspaces=false,
  escapeinside={(*@}{@*)},
  belowskip=2em,
  aboveskip=1em,
}


\newenvironment{answerfont}{\fontfamily{qhv}\selectfont}{\par}
\newenvironment{myanswer}{\begin{mdframed}\begin{answerfont}}{\end{answerfont}\end{mdframed}}


\title{HW 6}

\author{Mike Hanling}

\date{27 FEB 2017}

\begin{document}

\maketitle

\section*{Questions}
\label{sec:orgd8f1586}

\begin{enumerate}
\item (5 points) Why does the operating system perform system calls as
oppose to having each user perform the same operations
themselves?

\begin{myanswer}
System calls can be used to allocate memory.  If users were doing this
themselves, then there is a high chance that important information will
be overwritten due to negligence.
\end{myanswer}


\item (10 points) Look up the following C functions in the man page, label them as
either a system call or not a system call.

\begin{enumerate}
\item fread()

\begin{myanswer}
NOT
\end{myanswer}

\item write()

\begin{myanswer}
SYSTEM CALL
\end{myanswer}

\item stat()

\begin{myanswer}
SYSTEM CALL
\end{myanswer}


\item mmap()

\begin{myanswer}
SYSTEM CALL
\end{myanswer}


\item execv()

\begin{myanswer}
NOT
\end{myanswer}

\end{enumerate}

\item (10 points) Run \texttt{ic221-up}. In the \texttt{hw/06/prob3} directory you'll find a compiled
program. Use \texttt{ltrace} to enumerate the library function calls
occurring under \texttt{main()}.

\begin{myanswer}
3
strlen()
puts()
fflush()
\end{myanswer}

\item (10 points) Run \texttt{ic221-up}. In the \texttt{hw/06/prob4} directory you'll find a compiled
program. Use \texttt{strace} to enumerate the system function calls
occurring under \texttt{main()}.

\begin{myanswer}
Could not get executable to run correctly. Supposed to run
\texttt{strace ./trace-me-2 [arguments] > /dev/null} and then count all
of the system calls that will be listed with exit codes/return values.
\end{myanswer}

\item (20 points) Consider a file, \texttt{accts.dat}, which stores 1000 accounts
formatted based on the defined structure. Using \texttt{open()} and
\texttt{read()}, complete the program below to print them out:


\begin{lstlisting}[language=c]
#include <stdio.h>
#include <stdlib.h>
#include <unistd.h>
#include <fcntl.h>

typedef struct{
  long acctnum;
  double bal;
  char acctname[1024];
} acct_t;

void read_accts(accts){
  //COMPLETE ME
}

int main(int argc, char *argv[]){
  acct_t accts[1000];

  read_accts(accts);

  int i;
  for(i=0;i<1000;i++){
    printf("%ld (%f) -- %s\n",
           accts[i].acctnum,
           accts[i].bal,
           accts[i].acctname);
  }

  close(fd);

}
\end{lstlisting}

\begin{myanswer}
\begin{lstlisting}[language=c]
void read_accts(accts){
  char filename[128];
  printf("File of accounts: ");
  scanf("%s", filename);

  fd = open(filename, O_RDONLY);

  for (int i = 0; i < 1000; i++) {
    read(fd, &(acct[i].acctnum), sizeof(long));
    read(fd, &(acct[i].bal), sizeof(double));
    read(fd, &(acct[i].acctname), 1024);
  }

  //not needed in main
  close(fd);
}
\end{lstlisting}
\end{myanswer}

\item (10 points) Complete the following ORing options that matching the \texttt{fopen()} permissions:
\begin{enumerate}
\item \texttt{r}

\begin{myanswer}
O\_RDONLY
\end{myanswer}

\item \texttt{w}

\begin{myanswer}
O\_WRONLY \textbar O\_TRUNC \textbar O\_CREAT
\end{myanswer}

\item \texttt{a}

\begin{myanswer}
O\_WRONLY \textbar O\_APPEND \textbar O\_CREAT
\end{myanswer}

\item \texttt{w+}

\begin{myanswer}
O\_RDWR \textbar O\_CREAT \textbar O\_TRUNC
\end{myanswer}

\item \texttt{r+}

\begin{myanswer}
O\_RDWR
\end{myanswer}

\end{enumerate}

\item (10 points) Consider a \texttt{umask} of 0750 (the leading 0 is to indicate a number
written in octal). For the following \texttt{open()} permissions, what
actual permission will the file get? You can write your answers in
octal.

\begin{enumerate}
\item 0777

\begin{myanswer}
0000
\end{myanswer}

\item 0640

\begin{myanswer}
0137
\end{myanswer}

\item 0740

\begin{myanswer}
0037
\end{myanswer}

\item 0501

\begin{myanswer}
0276
\end{myanswer}

\item 0651

\begin{myanswer}
0126
\end{myanswer}
\end{enumerate}

\item (5 points) Explain why the umask is considered a security feature.

\begin{myanswer}
It ensures that even when a system call to \texttt{open()} requests
higher permissions than already set with \texttt{umask}, the OS will not
allow for those high permissions to be set.  It will default back the
highest available from the \texttt{umask}.
\end{myanswer}

\end{enumerate}
\end{document}
