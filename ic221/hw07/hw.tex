\documentclass{article}[9pt]

\usepackage{listings}
\usepackage{fullpage}
\usepackage{textcomp}
\usepackage{mdframed}

\lstset{ %
  basicstyle=\ttfamily\scriptsize,
  commentstyle=\ttfamily\scriptsize\emph,
  upquote=true,
  framerule=1.25pt,
  breaklines=true,
  showstringspaces=false,
  escapeinside={(*@}{@*)},
  belowskip=2em,
  aboveskip=1em,
}


\newenvironment{answerfont}{\fontfamily{qhv}\selectfont}{\par}
\newenvironment{myanswer}{\begin{mdframed}\begin{answerfont}}{\end{answerfont}\end{mdframed}}


\title{HW 7}

\author{Mike Hanling}

\date{3 MAR 2017}

\begin{document}

\maketitle

\section*{Questions}
\label{sec:orgfce6862}

\begin{enumerate}
\item (10 points) What is the difference between \texttt{\_exit()} and \texttt{exit()} and \texttt{\_Exit()}?

  \begin{myanswer}
    \texttt{\_exit()} is the system call that exits out of a process\\
    \texttt{exit()} is a library call that cleans up the current process
    and then calls \texttt{\_exit()}\\\
    \texttt{\_Exit()} is a library call that is simply a wrapper for the
    sys call \texttt{\_exit()} (no clean up)
  \end{myanswer}

\item (5 points) When a process successfully returns from \texttt{main()}, which of the
three different exit calls is actually used? What exit value is
typically used for the process when it returns from \texttt{main()} and
why?

  \begin{myanswer}
    A return from main calls \texttt{exit()} with an exit value of 0
  \end{myanswer}

\item (5 points) What is the difference between unbuffered, line buffered, and
fully buffered with respect to output streams?

  \begin{myanswer}
    Unbuffered: Every write call will immediately write to its dest
    Line Buffered: Writes will be commited once a new line character is
    supposed to be printed, or when the buffer is full\\
    Fully Buffered: The contents of the buffer will only be written once
    the buffer is completely filled
  \end{myanswer}

\item (20 points) Consider the following program snippets. What are the outputs of
each? \textbf{Explain your answer!}

\begin{enumerate}
\item
\begin{verbatim}
int main(){
  fprintf(stdout, "Hello World!");
  return 0;
}
\end{verbatim}

  \begin{myanswer}
    "Hello World!"
  \end{myanswer}

\item
\begin{verbatim}
int main(){
  fprintf(stdout, "Hello World!");
  exit(0);
}
\end{verbatim}

  \begin{myanswer}
    "Hello World!"
  \end{myanswer}

\item
\begin{verbatim}
int main(){
  fprintf(stdout, "Hello World!");
  _Exit(0);
}
\end{verbatim}

  \begin{myanswer}
    nothing
  \end{myanswer}

\item
\begin{verbatim}
int main(){
  fprintf(stderr, "Hello World!");
  _exit(0);
}
\end{verbatim}
  
  \begin{myanswer}
    nothing
  \end{myanswer}

\end{enumerate}

\item (10 point) Why does the following code snippet properly check for a failed
call to \texttt{execv()}?

\begin{verbatim}
int main(){
  char * ls_args[2] = { "/bin/ls", NULL} ;

  execv( ls_args[0], ls_args);
  perror("execve failed");

  exit(1); //failure
}
\end{verbatim}

  \begin{myanswer}
    If the call to \texttt{execv} runs smoothly, then nothing
    sequentially after that call in the C program will execute because
    the processes are completely swapped out.  Thus, having the call to
    \texttt{perror} afterwards will handle the possibility of the
    \texttt{ls} throwing an error.
  \end{myanswer}

\item (10 points) Consider setting up an \texttt{argv} array to be passed to execv() for
the execution of following command: 

\begin{verbatim}
ls –l –a /bin /usr/bin Fill in
\end{verbatim}

Complete the \texttt{argv} deceleration in code

\begin{verbatim}
char * argv[] = { /* what goes here? */ } ;
\end{verbatim}

  \begin{myanswer}
    \texttt{argv[1]} given that the program takes no other arguments
  \end{myanswer}

\item (5 points) The \texttt{fork()} system call is the only function that returns
\emph{twice}. Explain why this is?

  \begin{myanswer}
    It returns once as the parent and once as the parent
  \end{myanswer}

\item (5 points) If you were to compile and run the following program in the
shell, which process'es \texttt{pid} would print to the screen?
\textbf{Explain}

\begin{verbatim}
int main(){
  printf("Parent pid: %d\n", getppid());
}
\end{verbatim}

  \begin{myanswer}
    It will return the pid of the kernel.  The \texttt{getppid()} system
    call gets the pid of the current process's parent.  The parent of
    the main is the kernel.
  \end{myanswer}

  
\item (5 points) The \texttt{wait()} system call will return when a child's status change
of a child. What is the most typical status change that would
make the system call return?

  \begin{myanswer}
    Termination
  \end{myanswer}

\item (15 points) Using the manual page, provide a brief description of each of
the status macros below:


\begin{enumerate}
\item \texttt{WIFEXITED()}

  \begin{myanswer}
    Returns true if the child terminated normally (exit(3) or \_exit(2)
    or return from main)
  \end{myanswer}

\item \texttt{WIFEXITSTATUS()}

  \begin{myanswer}
    Does not exist.  Notes and my man page says WEXITSTAUS().  Returns
    the exit status of the child. Should only be employed if WIFEXITED()
    returns true
  \end{myanswer}

\item \texttt{WIFSIGNALED()}

  \begin{myanswer}
    Returns true id the child process was terminated by a signal
  \end{myanswer}

\end{enumerate}

\item (10 points) Assume you were writing a program that checked if a file
existed by using \texttt{ls}. (This is a silly way to do this, but just
for the sake of argument)

Recall that \texttt{ls} returns an exit status of 2 when the file does
not exist and it cannot list it, and \texttt{ls} returns an exit status
of 0 when the file does exist and can be listed. Complete the
\texttt{wait()} portion of the program below. The output should be
EXISTS! if the file specified in \texttt{argv[1]} exists and DOES NOT
EXIST! If the file specified in \texttt{argv[1]} does not exist. 

( \emph{hint: actually try and complete the program on your computer} )

\begin{verbatim}
#include <unistd.h>
#include <stdlib.h>
#include <stdio.h>
#include <sys/wait.h>
#include <sys/types.h>

int main(int argc, char * argv[]){
  pid_t cid;
  char * ls_args[] = {"ls", NULL, NULL};
  if(argc == 2){
    ls_args[1] = argv[1];
  }
  cid = fork();
  if( cid == 0 ){ /*child*/
    execvp(ls_args[0],ls_args);
    exit(1); /*error*/
  }

  /*parent*/
  int status;
  wait(&status);

  /* FINISH THIS PROGRA */

}
\end{verbatim}

  \begin{myanswer}
    \begin{lstlisting}[language=c]
    if(WIFEXITED(status)) {
      if(WEXITSTATUS(status) == 0)
        printf("EXISTS!");
      if(WEXITSTATUS(status) == 2)
        printf("DOES NOT EXIST!");
    }

    return 0;
    \end{lstlisting}
  \end{myanswer}

\end{enumerate}
\end{document}
%%% Local Variables:
%%% mode: latex
%%% TeX-master: t
%%% End:
