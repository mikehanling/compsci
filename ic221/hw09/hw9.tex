\documentclass{article}[9pt]
\usepackage{listings}
\usepackage{fullpage}
\usepackage{textcomp}
\usepackage{mdframed}
\lstset{ %
basicstyle=\ttfamily\scriptsize,
commentstyle=\ttfamily\scriptsize\emph,
upquote=true,
framerule=1.25pt,
breaklines=true,
showstringspaces=false,
escapeinside={(*@}{@*)},
belowskip=2em,
aboveskip=1em,
}\newenvironment{answerfont}{\fontfamily{qhv}\selectfont}{\par}

\newenvironment{myanswer}{\begin{mdframed}\begin{answerfont}}{\end{answerfont}\end{mdframed}}
\title{HW 9}
\author{Mike Hanling}
\date{02 APR 18}
\begin{document}\maketitle
\section*{Questions}
\label{sec:orgfce6862}
\begin{enumerate}


\item (5 points) What does it mean that signals arrive
asynchronously?
\begin{myanswer}
They can come from different processes, things do not run one after
another, signals can come from anywhere at anytime.
\end{myanswer}

\item (10 points) What signal is generated from the following keyboard-shortcut/command?

\begin{enumerate}
\item Ctrl-c
\begin{myanswer}
SIGINT
\end{myanswer}

\item Ctrl-z
\begin{myanswer}
SIGSTOP
\end{myanswer}

\item fg/bg
\begin{myanswer}
SIGCONT
\end{myanswer}

\item Ctrl-\
\begin{myanswer}
SIGQUIT
\end{myanswer}

\end{enumerate}
\item (5 points) Run the command kill -l to list all the signals and their
signal numbers. Find either the matching
signal-number/signal-name for the following values
below. Additionally, for each signal, use man 7 signal to
describe the default action of each.


\begin{enumerate}
\item SIGKILL
\begin{myanswer}
9: Terminate
\end{myanswer}

\item 14
\begin{myanswer}
SIGALRM: Terminate
\end{myanswer}

\item SIGALRM
\begin{myanswer}
14: Terminate
\end{myanswer}

\item SIGABRT
\begin{myanswer}
6: Terminate and core dump
\end{myanswer}

\item 21
\begin{myanswer}
SIGTTIN: Stop
\end{myanswer}

\item 1
\begin{myanswer}
SIGHUP: Terminate
\end{myanswer}

\item SIGCHLD
\begin{myanswer}
17: Ignore
\end{myanswer}

\item SIGTTOU
\begin{myanswer}
22: Stop
\end{myanswer}

\end{enumerate}
\item (15 points) Provide a brief plain-English explanation, e.g., reference which
signal is sent and to which process(es), for each the
kill/killall commands. (hint: You should look in the man
pages…)


\begin{enumerate}
\item killall -17 sleep
\begin{myanswer}
SIGCHLD signal sent to all sleep processes running
\end{myanswer}

\item kill -9 2237
\begin{myanswer}
SIGKILL sent to process 2237
\end{myanswer}

\item killall -SIGUSR1 a.out
\begin{myanswer}
Signal 10 is sent to all a.out processes
\end{myanswer}

\item kill -SIGABRT -1
\begin{myanswer}
Signal 6 sent to all process excluding kill and init
\end{myanswer}

\item killall sleep
\begin{myanswer}
SIGTERM sent to all sleep processes
\end{myanswer}

\item killall -u
\begin{myanswer}
Kill only processes owned by this user
\end{myanswer}

\end{enumerate}
\item 
On a lab machine, run the following program in the background


~aviv/ic221-hw/hw09/ic221-signaler \&


From the same (or another) terminal on the same machine, send
the program the following two signals below and describe the
results.





\begin{enumerate}
\item (3 points) SIGUSR1
\begin{myanswer}
"Ha Ha, missed me!"\\
An ASCII art cat.
\end{myanswer}

\item (3 points) SIGUSR2
\begin{myanswer}
"You shot me!"\\
ASCII art skull and crossed bones.
\end{myanswer}

\end{enumerate}
\item 
Consider the program below and answer the associated questions:



\begin{lstlisting}
int count = 0;

void handler(int signum){

  printf("You Shot Me!\n");
  count++;

  if(count > 3){
    printf("I'm dead!\n");
    exit(1);
  }

}

int main(){

  //establish signal hander for SIGTINT and SIGSTOP
  signal(SIGTINT,handler);
  signal(SIGTSTP,handler);

  //loop forever
  while(1);

}

\end{lstlisting}







\begin{enumerate}
\item (3 points) What is the output of the program if the user hits Ctrl-c only once? Explain.
\begin{myanswer}
You Shot Me!\\
SIGINT is handler and count is less than 4, so it prints the above once.
\end{myanswer}

\item (3 points) What is the output of the program if the user hits Ctrl-c four times? Explain.
\begin{myanswer}
You Shot Me!\\
You Shot Me!\\
You Shot Me!\\
You Shot Me!\\
I'm dead!\\
SIGINT is handled four times, and then count is greater than 3, so the
last line is printed and the program exits.
\end{myanswer}

\item (3 points) What is the output of the program if the user hits Ctrl-c three times and Ctrl-Z once? Explain.
\begin{myanswer}
You Shot Me!\\
You Shot Me!\\
You Shot Me!\\
You Shot Me!\\
I'm dead!\\
SIGINT and SIGSTOP are handled with the same function, so the
explanation from the quesion above still applies.
\end{myanswer}

\end{enumerate}
\item (5 points) What does the system call pause() do? Yes, it pauses
the program, of course, but until when does the program stay
paused and why is it a useful command?
\begin{myanswer}
It says paused until there is a signal received to terminate, stop, or
one that has a way to be handled.  It is useful with alarms, as it will
ahve the program sleep until the alarm has gone off, sending the SIGALRM
so that it can be handled.
\end{myanswer}

\item 
(10 points) How many times does the program below print alarm? Explain



\begin{lstlisting}
int count = 10;

void handler(int sugnum){

  printf("Alarm!\n");
  count /= 2;
  alarm(count);

}

int main(){

  signal(SIGALRM, handler);
  alarm(count);

  while(1) pause();
}

\end{lstlisting}


\begin{myanswer}
4 because it will print after 10 seconds, then 5, seconds, then 2,
seconds, then 1 second, and then the alarm is for zero seconds, so it
will never go off again.
\end{myanswer}

\item 
(10 points) Convert the following use of signal() below to a
sigaction() call.



\begin{lstlisting}
signal(SIGUSR1, handler);

\end{lstlisting}


\begin{myanswer}
\begin{lstlisting}[language=c]
struct sigaction action;
action.sa_handler = handler;
sigaction( SIGUSR1, &action, NULL);
\end{lstlisting}
\end{myanswer}

\item (5 points) What sigaction flag is used to ensure that system
calls will be restarted when interrupted?
\begin{myanswer}
SA\textunderscore RESTART
\end{myanswer}

\item (5 point) Provide an example of why the read() system call
would need to be restarted due to a signal delivery.
\begin{myanswer}
If the siganl comes from alarm, like making sure the user is not taking
too long to put in an input, then the read() will not restart, so the
user will never be able to input text even if it appears that way in a
program.
\end{myanswer}

\item (5 points) Look in man 7 signal and kill -l and draw a
picture of your favorite signal. Be sure to clearly identify
it. (You will not use LaTex bonus points for hand drawing your
picture.)
\begin{myanswer}
\begin{verbatim}

______________________________________________________||_._._._._._._._.
\_____________________________________________________||_#_K_I_L_L_9_#_|
                                                      11

\end{verbatim}
\end{myanswer}

\end{enumerate}
\end{document}

%%% mode: latex
% TeX-master: t
%%% End:
