\documentclass{article}[9pt]
\usepackage{listings}
\usepackage{fullpage}
\usepackage{textcomp}
\usepackage{mdframed}
\lstset{ %
basicstyle=\ttfamily\scriptsize,
commentstyle=\ttfamily\scriptsize\emph,
upquote=true,
framerule=1.25pt,
breaklines=true,
showstringspaces=false,
escapeinside={(*@}{@*)},
belowskip=2em,
aboveskip=1em,
}\newenvironment{answerfont}{\fontfamily{qhv}\selectfont}{\par}

\newenvironment{myanswer}{\begin{mdframed}\begin{answerfont}}{\end{answerfont}\end{mdframed}}
\title{HW 10}
\author{Mike Hanling}
\date{09APR18}
\begin{document}
\maketitle
\section*{Questions}
\label{sec:orgfce6862}
\begin{enumerate}


\item (5 points) Why must there be a de-escalation of privilege when the login
program executes the shell for an authenticated user?
\begin{myanswer}
The getty program runs as root, and not all users have the sam
permissions as root.  Therefore, permissions need to be scaled back to
just those fo the authenticated user.
\end{myanswer}

\item (10 points) Consider the following program with the following permission
strings below, if you (as your username) were to run these
programs, what capabilities (group and user permissions) would
the executing program have?

\begin{enumerate}
\item -rwxr-x--x 1 aviv scs 8622 Mar 30 10:40 a.out
\begin{myanswer}
User: execute\\
Group: none
\end{myanswer}

\item -rwsr-x--x 1 aviv scs 8622 Mar 30 10:40 a.out
\begin{myanswer}
User: read/write/execute\\
Group: none
\end{myanswer}

\item -rwxr-s--x 1 aviv scs 8622 Mar 30 10:40 a.out
\begin{myanswer}
User: execute\\
Group: read/execute
\end{myanswer}

\item -rwsr-s--x 1 aviv scs 8622 Mar 30 10:40 a.out
\begin{myanswer}
User:read/write/execute\\
Group: read/execute
\end{myanswer}

\end{enumerate}
\item (5 points) What is the difference between the real and effective user
and group id of a running process?
\begin{myanswer}
The real user id is the actual identifier of the current user.\\
The effective user id is the identifier that the prrogram will run
under.
\end{myanswer}

\item (15 points) Provide a short, plain-English, description of each of the system
calls below:

\begin{enumerate}
\item getuid()
\begin{myanswer}
Get the real user id.
\end{myanswer}

\item getgid()
\begin{myanswer}
Get the real group id.
\end{myanswer}

\item geteuid()
\begin{myanswer}
Get the effective user id.
\end{myanswer}

\item getegid()
\begin{myanswer}
Get the effective group id.
\end{myanswer}

\item setuid(uid)
\begin{myanswer}
Set the real user id to uid.
\end{myanswer}

\item setgid(gid)
\begin{myanswer}
Set the real group id to gid.
\end{myanswer}

\item setreuid(uid,euid)
\begin{myanswer}
Sets the real user id to uid and the effective user id to euid.\\
Using -1 for either leaves it unchanged.
\end{myanswer}

\item setregid(gid,egid)
\begin{myanswer}
Sets the real group id to uid and the effective group id to euid.\\
Using -1 for either leaves it unchanged.
\end{myanswer}

\end{enumerate}
\item (10 points) Consider the following chmod statements, provide the
permission string, that is the permission string rwxrwxrwx
represents 777. Be careful about setbits.

\begin{enumerate}
\item chmod 6750 a.out
\begin{myanswer}
rwsr-s---
\end{myanswer}

\item chmod 4750 a.out
\begin{myanswer}
rwsr-x---
\end{myanswer}

\item chmod 2750 a.out
\begin{myanswer}
rwx-r-s---
\end{myanswer}

\end{enumerate}
\item (5 points) Suppose you are writing a setuid program, and you want
downgrade the effective permission of the program back to the
user who is executing the program. Provide one line of C that
can do that.
\begin{myanswer}
seteuid(getuid());
\end{myanswer}

\item (5 points) What does the library call system() do?
\begin{myanswer}
It does the fork-exec-wait cycle for us.  Place any command to run at
the command in quotes as the argument, and system() executes it.
\end{myanswer}

\item (5 points) Explain how the environment variable PATH is used to select
which program to execute when using execvp() or system() or in a shell?
\begin{myanswer}
The actual executable for the command is searched in all the directories
(in order) in the PATH variable.  Any time a command is run, the PATH
variable is used to find the executable.
\end{myanswer}

\item 
(10 points)  The following program has a (multiple!) security flaw, describe
how to exploit it. And, provide at least one way to change the
program to protect it from the attack you described?



\begin{lstlisting}
#include <stdio.h>
#include <stdlib.h>
int main(){
  system("cat sample.db | cut -d ',' -f 3 | sort  | uniq");
}

\end{lstlisting}


\begin{myanswer}
If the PATH variable is corrupted, then different executables can be run
if they have the same name, and are found earlier while searching
through PATH. One way to fix that is to use execv (not execvp) in the
classic fork-exec-wait cycle and specifically give the executable to be run.
\end{myanswer}

\item 
(10 points) The following program has two security flaws, describe them
and how to exploit them.



\begin{lstlisting}
int main(){                                 
  char cmd[1024];
  char fname[40];
  printf("Enter file name:\n");
  scanf("%s",fname);
  snprtinf(cmd,1024,"/bin/cat %s",fname);
  system(cmd);
}

\end{lstlisting}


\begin{myanswer}
Overflow on fname and not checking for bash code. Understanding how much
buffer is allotted to fname and then presenting more would then
overwrite into other variable, such as cmd.  Also, if bash code is used
as input properly (add a ; and write some bad stuff), then the program
has been taken over.  In addition, the PATH variable is not set with
setenv(), so a PATH attack could also take place.  
\end{myanswer}

\item (10 points) Describe a solution to each of the security flaws you
identified in the previous question.
\begin{myanswer}
When reading in fname, use bound checking the same way it was done for
cmd. Check for any bash symbols in the input that could be hazardous and
turn them to null.  Use setenv() in the beginning of the main to ensure
that proper directories are being checked for the executables.
\end{myanswer}

\item Consider yourself as a software developer designing a tool for your
organization that takes advantage of different UNIX system tools. As such,
you wish to make use of the system() and popen() calls to inter
operate with your tool and the standard UNIX tools. While the tool need
high privilege levels (e.g., to log users in, access different
information), individual users may need varying lesser privilege levels,
but not necessarily equal across users.\\
(5 points) Describe three potential ethical and legal impacts on your
organization (with respect to actions attackers could take) if your software was
designed insecurely.
\begin{myanswer}
Assuming I leave vulnerabilities open on purpose or find out about them
after the fact: \\
1. I could easily curl some executables, chmod them, and then PATH
attack any computer.\\
2. I could raise my permissions when executing and then rm -rf /,
getting rid of all that person's files.\\
3. I could raise permissions and use that to snoop on people/collect
info about them that I was not authorized to get.
\end{myanswer}
(5 points) Describe a three coding practices you can employ to reduce the ethical
and legal impacts of insecurity in your software.
\begin{myanswer}
1. I can ensure to use setenv() in my programs to make sure PATH attacks
will not happen.\\
2. I can always bounds check any input to suppress buffer overflow
attacks.\\
3. I can scrub all input of any bash-like characters/commands in order
to not allow for bash to be inadvertently run.
\end{myanswer}
\end{enumerate}
\end{document}

%%% mode: latex
% TeX-master: t
%%% End:
