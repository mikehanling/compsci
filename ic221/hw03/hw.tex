
\documentclass{article}[9pt]

\usepackage{listings}
\usepackage{fullpage}
\usepackage{textcomp}
\usepackage{mdframed}

\lstset{ %
  basicstyle=\ttfamily\small,
  commentstyle=\ttfamily\small\emph,
  upquote=true,
  framerule=1.25pt,
  breaklines=true,
  showstringspaces=false,
  escapeinside={(*@}{@*)},
  belowskip=2em,
  aboveskip=1em,
}


\newenvironment{answerfont}{\fontfamily{qhv}\selectfont}{\par}
\newenvironment{myanswer}{\begin{mdframed}\begin{answerfont}}{\end{answerfont}\end{mdframed}}

%\newcommand{\myanswer}[1]{\begin{mdframed}{\parbox{.85\linewidth}{\fontfamily{qhv}\selectfont{#1}}}\end{mdframed}}

\title{HW 3}

\author{Mike Hanling}

\date{30 JAN 2016}
\begin{document}

\maketitle

\begin{enumerate}
\item (5 points) Write a small C program that uses \texttt{sizeof()} to report
the size in byte of each the types listed below. (You don't need
to submit the program, just the sizes.) 

\begin{enumerate}
\item \texttt{int}

\begin{myanswer}
4
\end{myanswer}


\item \texttt{char}

\begin{myanswer}
1
\end{myanswer}


\item \texttt{int *}


\begin{myanswer}
8
\end{myanswer}

\item \texttt{float *}

\begin{myanswer}
8
\end{myanswer}

\item \texttt{char *}

\begin{myanswer}
8
\end{myanswer}

\item \texttt{short}

\begin{myanswer}
2
\end{myanswer}

\item \texttt{int **}

\begin{myanswer}
8
\end{myanswer}


\item \texttt{float}

\begin{myanswer}
4
\end{myanswer}

\item \texttt{double}

\begin{myanswer}
8
\end{myanswer}


\end{enumerate}

\item (5 Points) For the sizes above, why is it that all the pointer types, even
the double pointer, have the same size in bytes?


\begin{myanswer}
Each of them are pointers, therefore they are the same size.  What each
\textit{points to} may be different in length, but the actual pointer
size is constant.
\end{myanswer}


\item (10 points) Rewrite the following C++ code in C:

\begin{lstlisting}[language=c]
#include <stdio>
using namespace std;

int main(int argc, char *argv[]){

 int j=10;
 int k;

 cout << "Enter a number" << endl;
 cin >> k;

 cout << "Num+10: " << k + 10 << endl;
}
\end{lstlisting}

\begin{myanswer}
\begin{lstlisting}[language=c]
#include <stdio.h>

int main(int argc, char *argv[]){

 int j = 10;
 int k;

 printf("Enter a number\n");
 scanf("\%g", &k);|

 printf("Num+10: \%f", k+10);

}
\end{lstlisting}
\end{myanswer}



\item (10 points) Complete the program below to print "Go Navy" to a new file
called \texttt{gonavy.txt}, "Beat Army" to a \texttt{beatarmy.txt}, and "Crash
Airforce" to standard error. 

\begin{lstlisting}[language=c]
#include <stdio.h>
#include <stdlib.h>

int main(int argc, char * argv[]){

   FILE * gonavy, *beatarmy;
   
   //WRITE THE REST!

}
\end{lstlisting}

\begin{myanswer}
\begin{lstlisting}[language=c]
#include <stdio.h>
#include <stdlib.h>

int main(int argc, char * argv[]){

   FILE * gonavy, *beatarmy;

   gonavy = fopen("gonavy.txt", "w");
   beatarmy = fopen("beatarmy.txt", "w");

   fprintf(gonavy, "Go Navy");
   fprintf(beatarmy, "Beat Army");
   fprintf(stderr, "Crash Airforce");

   fclose(gonavy);
   fclose(beatarmy);
   return 0;
}
\end{lstlisting}

\end{myanswer}

\item (10 points) For the following program snippet below, there are at least
\textbf{five} errors. Enumerate them.

\begin{lstlisting}[language=c]
for( int i=0 ; i<5 , i--){
   printf(i)
}
\end{lstlisting}

\begin{myanswer}
\begin{enumerate}
  \item ; instead of , after \texttt{i<5}

  \item \textbf{INFINITE LOOP}, maybe change to \texttt{i++}

  \item missing first arg to \texttt{printf}: "\%d"

  \item missing , to separate args of \texttt{printf}

  \item missing ; after \texttt{printf}
\end{enumerate}
\end{myanswer}

\item (10 points) For the following code snippets, what is the output and explain
that output? (Yes, you should program and run these program to
determine the output!)

\begin{enumerate}
\item 

\begin{lstlisting}[language=c]
unsigned int i = 4294967295;
printf("%d\n",i);
\end{lstlisting}


\begin{myanswer}
Output: -1\\
Explanation: unsigned ints go from 0 to 65535.  Since this is out of
range it yields -1.
\end{myanswer}

\item 

\begin{lstlisting}[language=c]
int i = 3.1519;
printf("%d\n", i);
\end{lstlisting}


\begin{myanswer}
Output: 3\\
Explanation: forcing int will simply chop the decimal piece.
\end{myanswer}


\item 
\begin{lstlisting}[language=c]
int i = (int) 1.5 + 2.5 + 3.5 + 4.5
printf("%d\n",i);
\end{lstlisting}

\begin{myanswer}
Output: 11\\
Explanation: the \texttt{(int) 1.5 + ...} evaluates to \texttt{1 + ...} ; adding
the remaining floats, the result is 11.5; forcing it to be an int chops
the .5, leaving 11.
\end{myanswer}

\end{enumerate}

\item (10 points) Consider the program snippet below and the memory diagram representing
that programs state at MARK 0. Complete a stack diagram for each
of the remaining MARKS 1-4. 

\begin{lstlisting}[language=c]
int a=0,b=0,*p;
p = &b; /* (0) */

*p = 15; /* (1) */

a = b;

b = 25; /* (2) */

p = &a; /* (3) */

(*p)++; /* (4) */
\end{lstlisting}

Mark 0 Diagram
\begin{verbatim}
.----.----.
| a  |  0 | 
|----|----| 
| b  |  0 | <-.
|----|----|   |
| p  |  .-+---'
'----'----'
\end{verbatim}

\begin{myanswer}
Mark 1 Diagram
\begin{verbatim}
.----.----.
| a  |  0 | 
|----|----| 
| b  |15  | <-.
|----|----|   |
| p  |  .-+---'
'----'----'
\end{verbatim}

Mark 2 Diagram
\begin{verbatim}
.----.----.
| a  | 15 | 
|----|----| 
| b  | 25 | <-.
|----|----|   |
| p  |  .-+---'
'----'----'
\end{verbatim}

Mark 3 Diagram
\begin{verbatim}
.----.----.
| a  | 15 | <-.
|----|----|   |
| b  | 25 |   |
|----|----|   |
| p  |  .-+---'
'----'----'
\end{verbatim}

Mark 4 Diagram
\begin{verbatim}
.----.----.
| a  | 16 | <-.
|----|----|   |
| b  | 15 |   |
|----|----|   |
| p  |  .-+---'
'----'----'
\end{verbatim}
\end{myanswer}

\item (10 points) What is the values of the array after this code completes?

\begin{lstlisting}[language=c]
//statically declaring an array
int array[10] = {0,1,2,3,4,5,6,7,8,9};
int * p = array+3;

p[0]=2018;

//<--- Value of array here?
\end{lstlisting}

\begin{myanswer}
\{0,1,2,2018,4,5,6,7,8,9\}
\end{myanswer}

\item (10 points) You are trying to copy an array from to another, and you write the following code:

\begin{lstlisting}[language=c]
int a[10] = {0,1,2,3,4,5,6,7,8,9};
int b[10];

//copy from a to b
a=b;
\end{lstlisting}


\begin{enumerate}
\item Why is this code incorrect?


\begin{myanswer}
First, it would say b=a to copy a into b. Second, you cannot set array
types equal to each other.
\end{myanswer}


\item Write a corrected code segment by replacing the offensive part
of the code above to copy the values in \texttt{b} to \texttt{a}.

\begin{myanswer}
\begin{lstlisting}[language=c]
// THIS WILL GIVE RANDOM VALUES TO a because b is uninitialized!!!
for (int i = 0; i < 10; i++){
  a[i] = b[i];
}
\end{lstlisting}
\end{myanswer}

\end{enumerate}

\item (10 points) The program below has at least three things wrong
with them. Enumerate them and write the corrected code.

\begin{lstlisting}[language=c]
#include <stdlib.h>
int main( int argc, char * argv[]){
  file * stream;

  stream = open("file.txt", "r");

  fprintf(stream, "Hello World");

  return 0;
}
\end{lstlisting}

\begin{mdframed}
\begin{enumerate}
  \item \texttt{<stdio.h>} not \texttt{<stdlib.h>}
  \item \texttt{FILE} not \texttt{file}
  \item \texttt{fopen} not \texttt{open}
  \item \texttt{"w"} not \texttt{"r"} to be able to print to the stream
\end{enumerate}

\begin{lstlisting}[language=c]
#include <stdio.h>
int main( int argc, char * argv[]){
  FILE * stream;

  stream = fopen("file.txt", "w");

  fprintf(stream, "Hello World");

  return 0;
}
\end{lstlisting}

\end{mdframed}

\end{enumerate}

\end{document}
%%% Local Variables:
%%% mode: latex
%%% TeX-master: t
%%% End:
