\documentclass{article}[9pt]
\usepackage{listings}
\usepackage{fullpage}
\usepackage{textcomp}
\usepackage{mdframed}
\lstset{ %
basicstyle=\ttfamily\scriptsize,
commentstyle=\ttfamily\scriptsize\emph,
upquote=true,
framerule=1.25pt,
breaklines=true,
showstringspaces=false,
escapeinside={(*@}{@*)},
belowskip=2em,
aboveskip=1em,
}\newenvironment{answerfont}{\fontfamily{qhv}\selectfont}{\par}

\newenvironment{myanswer}{\begin{mdframed}\begin{answerfont}}{\end{answerfont}\end{mdframed}}
\title{HW 11}
\author{Mike Hanling}
\date{16APR18}
\begin{document}\maketitle
\section*{Questions}
\label{sec:orgfce6862}
\begin{enumerate}


\item (5 points) What are the two parts of a packet? Which stores the address and
what stores the data?
\begin{myanswer}
The Header stores the address and the Payload stores the data.
\end{myanswer}

\item (5 points) What does it mean for the Internet to be "packet switched"?
\begin{myanswer}
It means that at junctions (routers and devices) the devices know
exactly where to send the packet solely on the Header, no need to look
at the Payload. There are no pre-determined routes, but hops are
determined on the go.
\end{myanswer}

\item (5 points) What are the different layers of the protocol stack and what
purpose do they serve in delivering a packet?
\begin{myanswer}
The Application Layer is the start and end of a packet's life (get
requests and requests for data by the user).\\
The Transport Layer is for ports, to make sure the right process on the
computer gets the right information.\\
The Internet Layer is where packets hop across junctions to get from one
destination to another through networks.\\
The Link Layer allows for packets to be shared across networks. Ex:
Ethernet and Wifi.\\
The Physical Layer is for transmitting the 0's and 2's across the
medium, the actual information in the packets.
\end{myanswer}

\item (5 points) An Internet address (version 4) is stored in how many bytes?
\begin{myanswer}
4
\end{myanswer}

\item (5 points) A domain name, unlike an IP address, is more human usable, what
protocol enables domain names to be resolved into IP addresses?
\begin{myanswer}
DNS
\end{myanswer}

\item (15 points) Using the host command line tool, resolve the following domain
names to an IPv4 address(es). Indicate those that also have IPv6
addresses, and also, rerun host a few times, indicate which hosts
IP addressed changed on subsequent runs.

\begin{enumerate}
\item www.cis.upenn.edu
\begin{myanswer}
IP : 158.130.69.163\\
IPv6 : 2607:f470:8:64:5ea5::d
\end{myanswer}

\item www.cs.swarthmore.edu
\begin{myanswer}
IP : 130.58.68.137
\end{myanswer}

\item www.usna.edu
\begin{myanswer}
IP : 10.4.36.20
\end{myanswer}

\item facebook.com
\begin{myanswer}
IP : 31.13.69.228\\
IPv6 : 2a03:2880:f103:83:face:b00c:0:25de
\end{myanswer}

\item microsoft.com
\begin{myanswer}
IP : 191.239.213.197:104.40.211.35:104.43.195.251:23.100.122.175:23.96.52.53
\end{myanswer}

\end{enumerate}
\item (10 points) What is the purpose of the port address? How many bytes and what
C type would naturally store a port address?
\begin{myanswer}
To make sure that the right process on the computer gets the right
packets. It is an unsigned short (2 bytes) and is stored in sockaddr\_in as 
sin\_port.
\end{myanswer}

\item (10 points) TCP provides reliable data transmission, but at what cost? Why
might you want to use UDP over TCP?
\begin{myanswer}
TCP has acknowledgements and extra sending of data to make it more
reliable.  This can slow down the time it takes the send the same amount
of data (send to moving on).
\end{myanswer}

\item (20 points) For each of the descriptions below of a network type, indicate
the type that best matches that description. Options include:
struct in\_addr, in\_addr\_t, s\_addr, 
struct sockaddr
,

struct sockaddr
\_in, sin\_family, sin\_port, sin\_addr,

struct addrinfo
, ai\_family, ai\_addr.


\begin{enumerate}
\item Specifies the address type, e.g., AF\_INET, for the
addrinfo structure.
\begin{myanswer}
\texttt{int ai\_family}
\end{myanswer}

\item Specifies the addres type, e.g., AF\_INET, for the
sockaddr\_in structure.
\begin{myanswer}
\texttt{short sin\_family}
\end{myanswer}

\item A type defined as a uint32
\begin{myanswer}
\texttt{uint32\_t in\_addr\_t}
\end{myanswer}

\item A generic address structure for sockets
\begin{myanswer}
\texttt{struct sockaddr}
\end{myanswer}

\item A structure to store an IPv4 Internet address
\begin{myanswer}
\texttt{struct in\_addr}
\end{myanswer}

\item An unsigned short storing the port for a sockaddr\_in
\begin{myanswer}
\texttt{short sin\_port}
\end{myanswer}

\item Structure used to hint at IP addresses for resolving as well
as storing results.
\begin{myanswer}
\texttt{struct addrinfo}
\end{myanswer}

\item Member of the sockaddr\_in that stores the address
\begin{myanswer}
\texttt{struct in\_addr sin\_addr}
\end{myanswer}

\item The sole member of the in\_addr structure
\begin{myanswer}
\texttt{in\_addr\_t s\_addr}
\end{myanswer}

\item A generic socket address returned in a addrinfo
\begin{myanswer}
\texttt{struct sockaddr ai\_addr}
\end{myanswer}

\item A specific address structure for sockets to store IP, port pairs
\begin{myanswer}
\texttt{struct sockaddr\_in}
\end{myanswer}

\end{enumerate}
\item (5 points) The following functions are opposites, inet\_ntoa() and
inet\_aton(), what are their purposes? Provide a small
example.
\begin{myanswer}
These functions switch between a readable IP address (w/ dots) and how
it actually stored (as four unsigned bytes). In order to store a dotted
IP, you use inet\_aton(dotted) to convert it.  And then to get it back
in the dotted form, you use inet\_ntoa(structure stored).
\end{myanswer}

\item (5 points) When assign a port to a socket address, which of these two
conversion should you use and why? htnos() vs. nths().
\begin{myanswer}
htons() because it will convert the short that is the port number to the
\texttt{sin\_port} that is stored in the \texttt{saddr\_in}.
\end{myanswer}

\item 
(10 points) Consider setting the address 10.4.32.41 on port 22. Complete
the code below to do that.



\begin{lstlisting}
struct sockaddr\_in
\end{lstlisting}
 saddr;

//TODO: what assignments come next?


\begin{myanswer}
\begin{lstlisting}
struct sockaddr\_in saddr;
saddr.sin\_family = AF\_INET;
inet\_aton("10.4.32.41", &(saddr.sin\_addr));
saddr.sin\_port = htons(22);

\end{lstlisting}
\end{myanswer}

\end{enumerate}
\end{document}

%%% mode: latex
% TeX-master: t
%%% End:
