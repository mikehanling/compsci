
\documentclass{article}[9pt]

\usepackage{listings}
\usepackage{fullpage}
\usepackage{textcomp}

\lstset{ %
  basicstyle=\ttfamily\small,
  commentstyle=\ttfamily\small\emph,
  upquote=true,
  framerule=1.25pt,
  breaklines=true,
  showstringspaces=false,
  escapeinside={(*@}{@*)},
  belowskip=2em,
  aboveskip=1em,
}


\newcommand{\myanswer}[1]{\framebox[.9\linewidth]{\parbox{.9\linewidth}{\fontfamily{qhv}\selectfont{#1}}}}

\title{HW 1}

\date{17 JAN 2018}

\author{Mike Hanling}

\begin{document}

\maketitle

\begin{enumerate}

\item (10 points) What are the three components of a Unix system?
  Provide an example of the interaction from a user perspective down
  to the hardware and back up.

      \myanswer{User Space, Kernel Space, Hardware.  Ex: The user can
      use the {\tt whoami} command in the Shell.  The kernel will then
      interpret the command to find a certain memory address that holds
      the user's name.  The kernel will have to tell the hardware to do
      this lookup.  The hardware will do this and send back the answer
      (in binary of course) to the kernel.  The kernel can than
      interpret the machine language to produce the user's name back.}

\item (15 points) Consider the {\tt ls -l} output below, label the
  output appropriately:

\begin{lstlisting}
drwxr-xr-x 2 aviv scs 4096 2013-12-22 10:57 demo/
-rw-r--r-- 1 aviv scs 13454 2013-12-22 10:56 text.dat
\end{lstlisting}

      \myanswer{The first line of output: 
        \begin{description}
          \item Directory
          \item Read permissions for user, group, and global
          \item Write permissions for user only
          \item Execute permissions for user, group, and global
          \item Owned by aviv
          \item In group scs
          \item 4096 bits in size
          \item Last modified 12DEC2013 at 1057
          \item Named demo
        \end{description}
        
        The second line of output:
        \begin{description}
          \item File
          \item Read permissions for user, group, and global
          \item Write permissions for user only
          \item No execute permissions
          \item Owned by aviv
          \item In group scs
          \item 13454 bits in size
          \item Last modified 12DEC2013 at 1056
          \item Named text.dat
        \end{description}
        }
\item (15 points) For the following commands, determine in which bin directory they live by using which command on a lab machine

  \begin{enumerate}
    \item {\tt ls} 

      \myanswer{/bin/ls}

    \item {\tt which} 

      \myanswer{/usr/bin/which}

    \item {\tt tac} 

      \myanswer{/usr/bin/tac}

    \item {\tt grep} 

      \myanswer{/bin/grep}

    \item {\tt cut} 

      \myanswer{/usr/bin/cut}

    \item {\tt chmod} 

      \myanswer{/bin/chmod}

    \item {\tt head} 

      \myanswer{/usr/bin/head}

    \item {\tt mv} 

      \myanswer{/bin/mv}

  \end{enumerate}


\item (10 points) Look up the {\tt tac} command in the man
  pages. Describe its operations, and give an example usage.

      \myanswer{{\tt tac} concatenates and prints files in reverse.  It
      does the reverse of {\tt cat}.  Ex:\\
      {\tt cat nums.txt\\
      1\\
      2\\
      3\\
      tac nums.txt\\
      3\\
      2\\
      1\\}}

\item (15 points) What are the three guiding principles of the Unix
  design philosophy?

      \myanswer{\begin{itemize}
      \item Write programs to do \textit{one thing} \& do it well.
      \item Write programs to \textit{work together}.
      \item Write programs tho handle \textit{text streams}, because that
      is a universal interface.
      \end{itemize}}

\item (10 points) What are the primary purposes of standard input, output, and error for different programs?

      \myanswer{\begin{tabular}{cl}
      stdin & Input stream for reading anything typed in the terminal\\
      stdout & Output stream for printing anything to the terminal\\
      stderr & Separate output stream for errors in programs (not
      program output)
      \end{tabular}}

\item (15 points) Consider the following command line with redirects:

\begin{lstlisting}
grep PA < sample-db.csv 2> oops > sample-db.PA.csv
\end{lstlisting}

\begin{enumerate}

\item What is the output file?

      \myanswer{sample-db.PA.scv}

\item What is the input file?

      \myanswer{sample.db.scv}

\item What is the error file?

      \myanswer{oops}


\end{enumerate}

\item (10 points) Using a pipeline as an example, why is it necessary to have standard error and standard output?


      \myanswer{{\tt tac FILE\_YOU\_DO\_NOT\_HAVE\_READ\_PERMISSIONS\_FOR | cat
      something.txt - else.txt}\\
      Since you do not have the permissions to perform {\tt tac}, there
      will be an error message.  In order to not have {\tt cat} print
      out the error message in the middle of its output to standard
      output, there must be a standard error stream to keep error
      messages and program output separate.}

\end{enumerate}

\end{document}

%%% Local Variables:
%%% mode: latex
%%% TeX-master: t
%%% End:
