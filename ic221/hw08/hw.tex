\documentclass{article}[9pt]

\usepackage{listings}
\usepackage{fullpage}
\usepackage{textcomp}
\usepackage{mdframed}

\lstset{ %
  basicstyle=\ttfamily\scriptsize,
  commentstyle=\ttfamily\scriptsize\emph,
  upquote=true,
  framerule=1.25pt,
  breaklines=true,
  showstringspaces=false,
  escapeinside={(*@}{@*)},
  belowskip=2em,
  aboveskip=1em,
}


\newenvironment{answerfont}{\fontfamily{qhv}\selectfont}{\par}
\newenvironment{myanswer}{\begin{mdframed}\begin{answerfont}}{\end{answerfont}\end{mdframed}}


\title{HW 8}

\author{Mike Hanling}

\date{19 MAR 18}

\begin{document}

\maketitle
\section*{Questions}
\label{sec:org3ac93b9}

\begin{enumerate}
\item (3 points) What is a zombie and how are they created? AND, why are zombie
process a bad thing? (process zombies not human zombies)

  \begin{myanswer}
    A zombie is a process that has terminated but has not been wait()
    'ed on by a parent, therefore its information still exists and stays
    that way \"forever\". They are bad because they eat up
    brains....I mean memory.
  \end{myanswer}


\item (3 points) What is an orphan process? How are they created and who "adopts"
all orphans?

  \begin{myanswer}
    An orphan is a process that is still running, but its parent has
    terminated.  init adopts all orphan processes.
  \end{myanswer}

\item (5 points) How are process groups and jobs related in the shell?

  \begin{myanswer}
    A job is completed through however many processes it needs, all
    labeled undder the same process group id.
  \end{myanswer}

\item (4 points) How long with the following shell command  run for and why?

\begin{verbatim}
sleep 10 | sleep 20 | sleep 100 | sleep 30 | sleep 1
\end{verbatim}

  \begin{myanswer}
    This will run for 100 seconds because all of the sleep processes
    will run in parallel, so the one that takes the longest is the
    overall time: sleep 100 runs for 100 seconds.
  \end{myanswer}

\item (5 points) Explain the difference between sequential and parallel execution
of a command line?

  \begin{myanswer}
    Sequential execution is when processes are doen one after another
    (waiting for the end of one before the beginning of another).
    Parallel execution is when multiple processes are executed at the
    same time (all starting at once).
  \end{myanswer}

\item (8 points) For each of the system calls associated with process groupings,
provide a brief explanation of each.

\begin{enumerate}
\item \texttt{setpgrp()}

  \begin{myanswer}
    This will set the process group of the calling process to itself.
  \end{myanswer}

\item \texttt{setpgid()}

  \begin{myanswer}
    This will set the process group of the first arguemnt to the second
    argument.  Zero for the first argument means that the calling
    process should be used.  Zero for the second argument means to set
    the process group to the pid of the first argument.
  \end{myanswer}

\item \texttt{getpgrp()}

  \begin{myanswer}
    This will get the process group of the calling process.
  \end{myanswer}

\item \texttt{getpgid()}

  \begin{myanswer}
    This will get the process group of the argument.
  \end{myanswer}

\end{enumerate}

\item (10 points) For each of the system calls with arguments, briefly describe the
resulting action with respect to the calling process or target
process.

\begin{enumerate}
\item \texttt{getpgid(0)}

  \begin{myanswer}
    This will retrieve the process group of the calling process.
  \end{myanswer}

\item \texttt{setpgid(0,0)}

  \begin{myanswer}
    This will set the process group of the calling process to the pid of
    the calling process.
  \end{myanswer}

\item \texttt{setpgid(0,pgid)}

  \begin{myanswer}
    This will set the process group of the calling process to pgid.
  \end{myanswer}

\item \texttt{setpgid(pid, 0)}

  \begin{myanswer}
    This will set the process group of pid to the pid of the calling
    process.
  \end{myanswer}

\end{enumerate}

\item (10 points) Consider the following code snippet, what is the output and why?
(\emph{Hint: why not run it?})

\begin{verbatim}
int main(){
  pid_t cpid;

  cpid = fork();
  if(cpid == 0){

    setpgrp();
    if( getpid() == getpgrp()){
      printf("C: SAME PGID\n");
    }
    exit(0);

  }else if(cpid > 0){

    if(getpgid(cpid) == cpid){
      printf("P: SAME PGID\n");
    }else{
      printf("P: NOT SAME PGID\n");
    }

    wait();
    exit(0);
  }

  exit(1);
}
\end{verbatim}

  \begin{myanswer}
    P: NOT SAME PGID\\
    C: SAME PGID\\
    This is because the parent is executing before the child (at least
    in this run).  Therefore, the parent will see the cpid not matching
    its group, but in the child it sets its group and then checks it.
  \end{myanswer}


\item (12 point) Consider the following code snippet. If we were to run this
program in a terminal, will it be properly terminated by Ctrl-c?
If so, why? If not, why not?

\begin{verbatim}
int main(){
  pid_t cpid;

  cpid = fork();
  if( cpid == 0 ){
    setpgrp();
    while(1);
  }else if( cpid > 0 ){
    wait(NULL);
    exit(0);
  }
  exit(1); //fork failed
}
\end{verbatim}

  \begin{myanswer}
    Yes, this will terminate properly because all of the processes are
    still in the foreground process group.
  \end{myanswer}


\item (12 point) Consider the following code snippet with the open file
\texttt{fight.txt} containing the text " Go Navy! Beat Army!" (yes,
there are spaces in there).  What is the output of this program,
and why?

\begin{verbatim}
int main(){
  pid_t cpid;

  int fd = open("fight.txt",O_RDONLY);
  char buf[1024];

  cpid = fork();
  if( cpid == 0 ){
    read(fd, buf, 10);
    exit(0);
  }else if( cpid > 0 ){
    wait(NULL); /* wait for child*/

    read(fd,buf, 10);
    write(1, buf, 10);
    exit(0);
  }
  exit(1); //fork failed
}
\end{verbatim}

  \begin{myanswer}
    The output is "Beat Army!".  This is because The child reads the
    first 10 bytes (" Go Navy!") and the parent is waiting.  Once the
    child terminates, the parent the rest 10 bytes and prints it out.
  \end{myanswer}


\item (4 points) What does it mean to "widow" a pipe?

  \begin{myanswer}
    The read end is closed.
  \end{myanswer}


\item (12 points) Consider the following code snippet with the open file
\texttt{fight.txt} containing the text " Go Navy! Beat Army!" (yes,
there are spaces in there).  What is the output of this program,
and why?

\begin{verbatim}
int main(){
  int fd_in = open("fight.txt",O_RDONLY);

  int fd_out = open("output.txt",O_WRONLY | O_TRUNC | O_CREAT,0755);
  char buf[1024];

  close(0);
  dup2(fd_in,0);

  close(1);
  dup2(fd_out,1);

  while(scanf("%s",buf) != EOF){
    printf("%s\n",buf);
  }

  return 0;
}
\end{verbatim}

  \begin{myanswer}
    The output file has the text line separated.  This is because stdin
    and stdout were duplicated into the new file descriptors for this
    problem.
  \end{myanswer}


\item (12 points) What is the missing code in the program below such that the
child's write to \texttt{stdout} will be ready by the parent through
its \texttt{stdin}?

\begin{verbatim}
int main(){
  pid_t cpid;
  int pfd[2], n;
  char gonavy[] = "Go Navy!";
  char buffer[1024];

  pipe(pfd);

  cpid = fork();
  if( cpid == 0 ){

    /* What goes here? */

    write(1, gonavy,strlen(gonavy));
  }else if( cpid > 0 ){

    /* What goes here? */


    n = read(0, buffer, 1024);
    write(1,buffer,n);
  }

  exit(1);
}
\end{verbatim}

  \begin{myanswer}
    First slot: dup2(0, 1);\\
    Second slot: dup2(1, 0);
  \end{myanswer}


\end{enumerate}
\end{document}
